\documentclass[12pt, letterpaper]{article}

\title{Problem Definition}
\date{Feb. 20}
\author{Jacob Freedman, Scott Harris, Bryan Hayes, John Meyer}

\usepackage[autostyle, english=american]{csquotes}
\usepackage{mathtools}
\usepackage{amssymb}

\begin{document}

\maketitle

\section{Definition}

Magic the Gathering consists of a finite set $ C $ of distinct cards.
$ C $ can be partitioned into sets $ S_1, S_2, ..., S_n $, each called a Magic the Gathering \enquote{set}.
$ C $ can also be partitioned into two sets $ L $ and $ L_0 $, where $ L $ are \enquote{basic lands} and $ L_0 $ are not.

Recall: A \enquote{multiset} $ M $ is a 2-tuple $ (A, m) $
where $ A $ is a set and $ m: A \rightarrow \{x \in \mathbb{N} \mid x \ge 1 \} $.
$ A $ is called the \enquote{support} of $ M $ and
a set $ U $ is a \enquote{universe} of $ M $ if and only if $ A \subseteq U $.
The \enquote{sum} of multisets $ M_1 $ and $ M_2 $ is defined as
$ (A_1 \cup A_2, \{ x \in (A_1 \cup A_2) \mid (x, m_1(x) + m_2(x)) \} $.

Let $ P $ denote probability.

In a limited, sealed-deck tournament style, a partition $ S $ is chosen and used for play.
Each participant is provided a multiset $ B $ of cards whose universe is $ S $
as well as a multiset $ B_L $ whose universe is $ L $.
$ B $ is constrained to contain $ 90 $ cards
whereas $ B_L $ contains an infinite amount of cards.

Given participants must select a multiset $ D, \ |D| \ge 40 $ from the sum of $ B $ and $ B_L $,
we want to choose $ D $ such that we maximize $ P(\text{game success} \mid D) $.

\end{document}
