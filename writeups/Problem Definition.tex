\documentclass[12pt,letterpaper]{report}
\title{Problem Definition}
\date{Feb. 17}
\author{Jacob Freedman, Scott Harris, Bryan Hayes, John Meyer}

\usepackage[autostyle, english=american]{csquotes}

\begin{document}

\maketitle

\section{Introduction}

Magic the Gathering is a popular card game that has withstood the test of time,
boasting of twenty million players as of 2015.
Its sustained interest over the years has lead to organized tournaments
and has even attracted the interest of academic research and scholarly articles.
The game hinges on both strategy and uncertainty.
In a tournament style known as \enquote{limited, sealed-deck,}
players must pre-select 40 or more cards from a set of 90 cards given to them.
Players must consider the strengths and weaknesses of each card,
the synergies that may exist between them,
and forecast their utility during gameplay,
given that their opponent is doing the same.
We propose to help these participants by authoring an objective function for this
domain to score potential card combinations according to its strategic utility
as illuminated by common play-styles and professional strategies.
We also plan to implement an optimization algorithm to maximize this function
so that players can play with the most effective deck.
This algorithm could help professionals as well as novices to have better odds of success
and a more enjoyable experience.
Given Magic the Gathering’s large audience,
this algorithm would solve a practical problem common to players of the game
and could even see widespread use.

\section{Definition}

Magic the Gathering consists of a finite set $ C $ of distinct cards.
$ C $ can be partitioned into sets $ S_1, S_2, ..., S_n $, each called a Magic the Gathering \enquote{set}.
$ C $ can also be partitioned into two sets $ L $ and $ L_0 $, where $ L $ are "basic lands" and $ L_0 $ are not.

In a limited, sealed-deck tournament style,
a set $ S $ is chosen and used for play.

Each participant is provided with 6 booster packs consisting of $ 15 $ cards, totaling to $ 90 $ cards.
These cards $ B $ are a multiset whose elements are in $ S $.
Participants are also provided with an infinite mutliset $ B_L $ whose elements are in $ L $.

Participants must select a multiset $ D \subseteq B \cup B_L , \ |D| \ge 40 $.

We want to find $ D $ such that we maximize $ P(game success | D) $.

\end{document}
