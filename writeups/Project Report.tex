\documentclass[12pt, letterpaper]{article}

\title{Problem Definition}
\date{Feb. 20}
\author{Jacob Freedman, Scott Harris, Bryan Hayes, John Meyer}

\usepackage[autostyle, english=american]{csquotes}
\usepackage{mathtools}
\usepackage{amssymb}
\usepackage{setspace}

\doublespacing

\begin{document}

\maketitle

\section{Introduction}

Magic the Gathering is a popular card game that has withstood the test of time,
boasting of twenty million players as of 2015.
Its sustained interest over the years has lead to organized tournaments
and has even attracted the interest of academic research and scholarly articles.
The game hinges on both strategy and uncertainty.
In a tournament style known as \enquote{limited, sealed-deck,}
players must pre-select 40 or more cards from a set of 90 cards given to them.
Players must consider the strengths and weaknesses of each card,
the synergies that may exist between them,
and forecast their utility during gameplay,
given that their opponent is doing the same.
We propose to help these participants by authoring an objective function for this
domain to score potential card combinations according to its strategic utility
as illuminated by common play-styles and professional strategies.
We also plan to implement an optimization algorithm to maximize this function
so that players can play with the most effective deck.
This algorithm could help professionals as well as novices to have better odds of success
and a more enjoyable experience.
Given Magic the Gathering’s large audience,
this algorithm would solve a practical problem common to players of the game
and could even see widespread use.

\section{Definition}

Magic the Gathering consists of a finite set $ C $ of distinct cards.
$ C $ can be partitioned into sets $ S_1, S_2, ..., S_n $, each called a Magic the Gathering \enquote{set}.
$ C $ can also be partitioned into two sets $ L $ and $ L_0 $, where $ L $ are \enquote{basic lands} and $ L_0 $ are not.

Recall: A \enquote{multiset} $ M $ is a 2-tuple $ (A, m) $
where $ A $ is a set and $ m: A \rightarrow \{x \in \mathbb{N} \mid x \ge 1 \} $.
$ A $ is called the \enquote{support} of $ M $ and
a set $ U $ is a \enquote{universe} of $ M $ if and only if $ A \subseteq U $.
The \enquote{sum} of multisets $ M_1 $ and $ M_2 $ is defined as
$ (A_1 \cup A_2, \{ x \in (A_1 \cup A_2) \mid (x, m_1(x) + m_2(x)) \} $.

Let $ P $ denote probability.

In a limited, sealed-deck tournament style, a partition $ S $ is chosen and used for play.
Each participant is provided a multiset $ B $ of cards whose universe is $ S $
as well as a multiset $ B_L $ whose universe is $ L $.
$ B $ is constrained to contain $ 90 $ cards
whereas $ B_L $ contains an infinite amount of cards.

Given participants must select a multiset $ D, \ |D| \ge 40 $ from the sum of $ B $ and $ B_L $,
we want to choose $ D $ such that we maximize $ P(\text{game success} \mid D) $.

\section{Literature Review}

\section{Scott}

@article{Cowling:2012:10.1109/TCIAIG.2012.2204883,
author = {Peter I. Cowling, Colin D. Ward, Edward J. Powley},
title = {Ensemble Determinization in Monte Carlo Tree Search for the Imperfect Information Card Game Magic: The Gathering},
journal = {IEEE Transactions on Computational Intelligence and AI in Games},
issue_date = {June 2012},
volume = {4},
number = {4},
month = Dec,
year = {2012},
issn = {1943-0698},
pages = {241--257},
numpages = {16},
url = {https://ieeexplore.ieee.org/document/6218176},
doi = {10.1109/TCIAIG.2012.2204883},
publisher = {IEEE},
}

This article discusses how the Monte Carlo Tree Search (MCTS) algorithm has been applied to create AI’s for games in the
past and how it can be applied with the game Magic: the Gathering. Due to Magic’s core element of having incomplete
information while playing the game due to the fact that both players have hidden hands of cards that were drawn from
shuffled decks makes the Monte Carlo Tree Search approach quite effective. In addition, since each player gets to
create their own deck to play in the manner of their choice, this adds further hidden information as well as an
interesting challenge to the game. Other key elements of Magic: the Gathering such as the ability of a player to
interrupt another player during their turn, interact with other player’s resources, and predicting how an opponent
react to different plays makes this game an interesting field of study. The MCTS algorithm provides the advantage over
other algorithms since it is not required for a terminal game state to be reached for a result to be calculated but
instead can be stopped at any point to evaluate the current state. This paper details all of the core rules of Magic
such that they compare the advantages of the MCTS algorithm over a strictly rule based approach. By pruning different
kinds of moves to be added to the tree, the ability to formulate the best move is much more efficient. By using multiple
trees, the authors compensate for incomplete information allowing for the ideal course of action based on the known
information.


Ham, Ethan. “Rarity and Power: Balance in Collectible Object Games.” Game Studies - Narrative, Games, and Theory, 2010,
gamestudies.org/1001/articles/ham.

Mr. Ham details the issues that come along with the collectable card game format. The main issue involves the game both
being fun as well as balanced. In the game Magic: the Gathering, there were 9 cards within the first sets that were
significantly more powerful than their counterparts. The game designers did not intend for these cards to so
significantly impact the game that each player would own and use two or so copies of these cards in roughly every deck.
By these cards being so overpowered, the balancing of the game was completely out of whack. Other games such as Sanctum
have tried their hardest to not create cards that were so good that they could individually break the gameplay of
Sanctum, especially if the reason they were so good is that they were hard to obtain. These applications also apply to
trading baseball card games. In the game you can play with baseball cards, all of the baseball players all can be
reduced to a statistical value. This allows for many different strategies such that a player can accomplish the same
goals with many different strategies/collections of cards. With this versatility Suitcase players are not put at as much
of an advantage in the baseball card game as they are in games such as Magic: the Gathering.

\section{Jacob}

Martin Pelikan, David Goldburg, Erick Cantú-Paz. 1999. BOA: Bayesian Optimization Algorithm. Illinois Genetic Algorithms
Laboratory. University of Illinois at Urbana-Champaign,Urbana IL.

This paper describes how a bayesian optimization algorithm can be used to increase the efficiency of genetic algorithms.
It describes how genetic algorithms are optimization programs based on artificial selection and genetic recombination
operators and how the two most important factors to a genetic algorithms success, proper growth and mixing, are often
not achieved. The paper describes how a bayesian algorithm functions. It selects the best options from a population
using any searching method. Than the new options are added to the old population, replacing some of the old options.
This generates a bayesian network, where each node is one variable. This method could be helpful for creating our
optimization algorithm because it is fast. As shown in the paper, the algorithm runs in close to linear time. This
method could also be useful because it selects the best from a network, which would be an optimal way of selecting the
best cards for the deck.

Marie-Liesse Cauwet Mines Saint-Etienne, Olivier Teytaud. 2018. Surprising strategies obtained by stochastic
optimization in partially observable games. Univ Clermont Auvergne, CNRS, UMR 6158 LIMOS, Institut Henri Fayol,
Departement GMI, France

This paper takes a different approach to optimization. The authors use a evolutionary algorithm and a stochastic
optimization algorithm to find optimal outputs for partially observable games. They use 5 different algorithms, one
naive evolutionary algorithm, two coevolutionary algorithms, a iterative evolutionary algorithm, and a seed method.
These algorithms are tested by playing multiple different games, batawaf, war, cheat, battleship and guess who. The
results found that these algorithms found strategies for games that seem entirely luck based. This type of algorithm
could be useful for finding the optimal deck against different styles of play.

\end{document}
