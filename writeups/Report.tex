\documentclass[12pt, letterpaper]{article}

\title{Problem Definition}
\date{April 30}
\author{Jacob Freedman, Scott Harris, Bryan Hayes, John Meyer}

\usepackage[autostyle, english=american]{csquotes}
\usepackage{mathtools}
\usepackage{amssymb}
\usepackage{setspace}
\usepackage{listings}
\usepackage{commath}
\usepackage{amsmath}
\usepackage{minted}

\usepackage[colorlinks=true,linkcolor=blue,urlcolor=black,bookmarksopen=true]{hyperref}

\doublespacing

% Define subsections down to \subsubsubsection
\usepackage{titlesec}
\setcounter{secnumdepth}{4}

\titleclass{\subsubsubsection}{straight}[\subsection]
\newcounter{subsubsubsection}[subsubsection]

\renewcommand\thesubsubsubsection{\thesubsubsection.\arabic{subsubsubsection}}
\renewcommand\theparagraph{\thesubsubsubsection.\arabic{paragraph}}

\titleformat{\subsubsubsection}
{\normalfont\normalsize\bfseries}
{\thesubsubsubsection}
{1em}
{}

\titlespacing*{\subsubsubsection}
{0pt}
{3.25ex plus 1ex minus .2ex}
{1.5ex plus .2ex}

\begin{document}

\maketitle

\section{Introduction}

Magic the Gathering is a popular card game that has withstood the test of time,
boasting of twenty million players as of 2015.
Its sustained interest over the years has lead to organized tournaments
and has even attracted the interest of academic research and scholarly articles.
The game hinges on both strategy and uncertainty.
In a tournament style known as \enquote{limited, sealed-deck,}
players must pre-select 40 or more cards from a set of 90 cards given to them.
Players must consider the strengths and weaknesses of each card,
the synergies that may exist between them,
and forecast their utility during gameplay,
given that their opponent is doing the same.
We propose to help these participants by authoring an objective function for this
domain to score potential card combinations according to its strategic utility
as illuminated by common play-styles and professional strategies.
We also plan to implement an optimization algorithm to maximize this function
so that players can play with the most effective deck.
This algorithm could help professionals as well as novices to have better odds of success
and a more enjoyable experience.

\section{Definition}

Magic the Gathering consists of a finite set $ C $ of distinct cards.
$ C $ can be partitioned into sets $ S_1, S_2, ..., S_n $, each called a Magic the Gathering \enquote{set}.
$ C $ can also be partitioned into two sets $ L $ and $ L_0 $, where $ L $ are \enquote{basic lands} and $ L_0 $ are not.

Recall: A \enquote{multiset} $ M $ is a 2-tuple $ (A, m) $
where $ A $ is a set and $ m: A \rightarrow \{x \in \mathbb{N} \mid x \ge 1 \} $.
$ A $ is called the \enquote{support} of $ M $ and
a set $ U $ is a \enquote{universe} of $ M $ if and only if $ A \subseteq U $.
The \enquote{sum} of multisets $ M_1 $ and $ M_2 $ is defined as
$ (A_1 \cup A_2, \{ x \in (A_1 \cup A_2) \mid (x, m_1(x) + m_2(x)) \} $.

Let $ P $ denote probability.

In a limited, sealed-deck tournament style, a partition $ S $ is chosen and used for play.
Each participant is provided a multiset $ B $ of cards whose universe is $ S $
as well as a multiset $ B_L $ whose universe is $ L $.
$ B $ is constrained to contain $ 90 $ cards
whereas $ B_L $ contains an infinite amount of cards.

Given participants must select a multiset $ D, \ |D| \ge 40 $ from the sum of $ B $ and $ B_L $,
we want to choose $ D $ such that we maximize $ P(\text{game success} \mid D) $.

\section{Literature Review}

\subsection{\enquote{Ensemble Determinization in Monte Carlo Tree Search for the Imperfect Information Card Game Magic: The Gathering.}}
\label{sec:3.1}

This article discusses how the Monte Carlo Tree Search (MCTS) algorithm
has been applied to create AI's for games in the
past and how it can be applied with the game Magic: the Gathering.
Due to Magic's core element of having incomplete information while playing the game
due to the fact that both players have hidden hands of cards that were drawn from
shuffled decks makes the Monte Carlo Tree Search approach quite effective.
In addition,
since each player gets to create their own deck to play in the manner of their choice,
this adds further hidden information as well as an interesting challenge to the game.
Other key elements of Magic: the Gathering such as the ability of a player to
interrupt another player during their turn,
interact with other player's resources, and predicting how an opponent
react to different plays makes this game an interesting field of study.
The MCTS algorithm provides the advantage over
other algorithms since it is not required for a terminal game state to be reached for a result to be calculated
but instead can be stopped at any point to evaluate the current state.
This paper details all of the core rules of Magic such that they compare the advantages of the
MCTS algorithm over a strictly rule based approach.
By pruning different kinds of moves to be added to the tree,
the ability to formulate the best move is much more efficient.
By using multiple trees,
the authors compensate for incomplete information allowing for the ideal course of action based on the known information.

\subsection{\enquote{Rarity and Power: Balance in Collectible Object Games.}}
\label{sec:3.2}

Mr. Ham details the issues that come along with the collectable card game format.
The main issue involves the game both being fun as well as balanced.
In the game Magic: the Gathering,
there were 9 cards within the first sets that were significantly more powerful than their counterparts.
The game designers did not intend for these cards to so
significantly impact the game that each player would own and use two or so copies of these cards in roughly every deck.
By these cards being so overpowered, the balancing of the game was completely out of whack.
Other games such as Sanctum have tried their hardest to not create cards
that were so good that they could individually break the gameplay of
Sanctum, especially if the reason they were so good is that they were hard to obtain.
These applications also apply to trading baseball card games.
In the game you can play with baseball cards, all of the baseball players all can be reduced to a statistical value.
This allows for many different strategies such that a player can accomplish the same
goals with many different strategies/collections of cards.
With this versatility Suitcase players are not put at as much
of an advantage in the baseball card game as they are in games such as Magic: the Gathering.

\subsection{\enquote{BOA: Bayesian Optimization Algorithm.}}
\label{sec:3.3}

This paper describes how a bayesian optimization algorithm can be used to increase the efficiency of genetic algorithms.
It describes how genetic algorithms are optimization programs
based on artificial selection and genetic recombination operators
and how the two most important factors to a genetic algorithms success, proper growth and mixing, are often not achieved.
The paper describes how a bayesian algorithm functions.
It selects the best options from a population using any searching method.
Than the new options are added to the old population, replacing some of the old options.
This generates a bayesian network, where each node is one variable.
This method could be helpful for creating our optimization algorithm because it is fast.
As shown in the paper, the algorithm runs in close to linear time.
This method could also be useful because it selects the best from a network,
which would be an optimal way of selecting the best cards for the deck.

\subsection{\enquote{Surprising strategies obtained by stochastic optimization in partially observable games.}}
\label{sec:3.4}

This paper takes a different approach to optimization.
The authors use a evolutionary algorithm and a stochastic optimization algorithm
to find optimal outputs for partially observable games.
They use 5 different algorithms, one naïve evolutionary algorithm, two coevolutionary algorithms,
an iterative evolutionary algorithm, and a seed method.
These algorithms are tested by playing multiple different games, batawaf, war, cheat, battleship and guess who.
The results found that these algorithms found strategies for games that seem entirely luck based.
This type of algorithm could be useful for finding the optimal deck against different styles of play.

\subsection{\enquote{Active Set Methods with Reoptimization for Convex Quadratic Integer Programming}}
\label{sec:3.5}

\enquote{Active Set Methods with Reoptimization for Convex Quadratic Integer Programming}
proposes a new algorithm to solve \enquote{unconstrained convex quadratic integer}
problems with linear constraints.
The algorithm they propose is a generalized branch-and-bound approach,
which is the most effective method to solve problems of this type (quadratic integer programming problems),
and it has wide application as many real-world problems can be represented in this fashion.
It consists of a preprocessing phase and an iterative main phase,
which uses an optimization known as a \enquote{warmstart.}
For this optimization,
each iteration of the main phase uses knowledge of the previous iteration to inform its next guess.
If the number of constraints are held constant,
the algorithm runs in linear time with respect to the problem size.
It was tested with 200 random problem instances and it was found to outperform IBM CPLEX by a moderate measure.

\subsection{\enquote{Approximating the k-Set Packing Problem by Local Improvements}}
\label{sec:3.6}

\enquote{Approximating the k-Set Packing Problem by Local Improvements}
proposes a more efficient variant of many algorithms already created to solve the k-set packing problem.
The k-set packing problem, given a universe $ U $ and a collection $ S \subset U $ with $ |S| \le k $,
seeks to find the maximum number of sets $ \subset S $ which are mutually disjoint.
Prior work has improved solving this problem from $ \frac{k}{2} $ to $ \frac{k + 2}{3} $
then $ \frac{k + 1}{3} + \epsilon $ and $ \frac{1}{\epsilon} $.
The changes they propose further reduce it to $ \frac{1}{\epsilon^2} $.
They suggest that their contribution is possibly the best improvement that can be made for the problem,
since it matches a theoretical lower bound.
They use a local search strategy,
which is the most widely used approach for this problem.
Their algorithm, which they call \enquote{Algorithm CITC,}
uses a color-coding technique to make a logarithmic-time improvement with a polynomial-time algorithm.
They construct a \enquote{bipartite conflict graph}
and search it for what they call \enquote{colorful tail changes} for every $ 3 \ge $ degree vertex.
For each one they find,
they build a multigraph and continue computation.
To conclude, they use various strategies to squeeze out better performance from an already efficient solution
and do so perhaps for the last time.

\subsection{\enquote{Magic: The Gathering, A Literary Text.}}
\label{sec:3.7}

This reading starts off by talking about Magic: The Gathering and a bit of its history.
It compares the card game to the popular role-playing game
Dungeons \& Dragons in terms of its lore and overall aesthetic,
then goes on to talk about its impact on pop culture.
After this, the reading explains some of the rules of Magic,
and some of the different ways the game is played such as arranging the deck based on color of the card or
limiting the total size of the deck.
The main argument of this article is then presented;
that Magic is a form of literary work.
The author is arguing that,
while each card on its own presents little context,
the cards together paint a picture of a world filled with rich lore.
The author backs this up using the fact that works of interactive fiction,
such as graphic novels or motion comics,
are themselves considered works of literature by today's standards.
It is argued that Magic can be viewed as a form of interactive fiction.
In addition to this, the point is brought up that the Magic: The Gathering website
has a large collection of written pieces about the characters and lore of the Magic mythos.
All of these arguments, and the fact that Magic is a wildly popular and acclaimed game,
lead the author to claim that the card game is a \enquote{dynamic literary text.}

\subsection{\enquote{The Complexity of Deciding Legality of a Single Step of Magic: The Gathering.}}
\label{sec:3.8}

The reading begins with a short introduction on Magic: The Gathering,
where it explains the general circumstances and goals of the game.
It then moves on to discuss the purpose of the reading;
deciding the complexity of the legality of a single move in Magic.
The general rules and types of cards and what do they are then explained in detail.
The concepts of the deck, the player's hand, the stack, the battlefield,
and the graveyard are briefly discussed, and the types of cards included are instants/sorceries,
artifacts/enchantments, planeswalkers, creatures, and lands.
Some of the variants of the game are discussed,
including one of the single player variants known as Goldfish.
It then moves on to discuss the mathematical implications of
deciding the time complexity of the legality of a single move.
The reading discusses the ideas of min-cost flow,
the base case of having no requirements or restrictions on a move,
and the types of possible restrictions.
These restrictions include local restrictions, local counting restrictions,
global counting restrictions, blocker capacity restrictions, and attack capacity restrictions.
The complexity of the base case is then discussed,
as well as the complexities of special cases.
The reading concludes with a more in depth analysis of restriction one, the local restriction.

\subsection{Literary Analysis Conclusion}

In these scholarly readings regarding the game Magic: The Gathering and optimization algorithms,
we have gained a greater understanding of both how the game functions
as well as how we can accomplish our goal of optimizing the creation of competitive decks
based on a set of randomly generated cards.
Some possible strategies that we have read about include
using a Monte Carlo Search Tree approach, genetic programming, as well as active set methods.

\section{Algorithm Design}

In order to determine the optimal deck of cards that can be created with a generated pool of cards,
we must define a way of evaluating each created deck such that an optimization algorithm can determine
the best subset of the card pool would perform most optimally.
For this we have defined an objective function that will analyze a proposed deck
and calculate the penalty level of the given deck for the optimization algorithm to work with.

\subsection{Objective Function}

In order to calculate the penalty level on a single deck,
we have identified 6 sub-objectives each evaluating a different
important aspect of successful decks that we want to consider in our analysis.
% Some of the specified penalty values have not yet been fully tested and may change in the future.

\subsubsection{Deck Size}
\label{objective:Deck Size}

In the sealed format of Magic: The Gathering,
players are given 6 booster packs, each containing 15 cards,
and are able to build a deck out of whichever cards they get out of the packs.
The player can also add as many basic lands as they desire,
which are used to activate and play other cards.
When it is all said and done,
the ideal number of cards to have in a deck is the minimum allowed, 40 cards.
The reasoning behind this is that, when drawing a card,
having a smaller deck size will increase odds of the player drawing a useful card.
The larger the deck is,
the more likely it is filled with \enquote{bloat} cards,
or cards that are not very useful.
As such we will penalize for excess cards using the following quadratic function:

$$
f(n) = (40 - n) ^ 2
$$

with $ n $ being the actual number of cards in a constructed deck.

\subsubsection{Mana Curve}
\label{objective:Mana Curve}

Magic: The Gathering cards require certain amounts of \enquote{mana} in order to be played.
This mana cost is denoted on the top-right corner of every card.
Ignoring the specific mana \enquote{color} required,
each card can be categorized by the amount of mana required to play it.
For example, a card with \{1\}\{R\} would have an overall cost of 2 mana
where as a card with just \{R\} would have an overall cost of 1.
This overall cost is known as the \enquote{converted mana cost} or CMC.

Looking at the converted mana cost can be particularly useful in evaluating a deck.
A card's strength is generally directly correlated to its converted mana cost.
So, 8-CMC cards tend to be very powerful
whereas 1-CMC cards tend to be not as strong.
However, since a player's mana pool only slowly increases throughout the game,
someone who stocks up on the more-powerful cards may not be able to play them until late in the game.
On the flip side, if a deck is primarily filled with weaker 1-CMC cards,
then he can be very active in the beginning of the game
but may lack the more-powerful cards needed for the end-game.
So there exists a fine balance between how many cards of each CMC a deck should have.
This balance is what players refer to as an ideal \enquote{mana curve}.

If one were to count the converted mana cost of each card in the deck and store it in a vector,
the relative frequency graph of that vector is the \enquote{mana curve}.
Specifically, the ideal mana curve can be represented with the following R code:

\begin{lstlisting}[language=R, frame=single]

ideal.mana.curve <- c(
    1,
    2, 2, 2, 2, 2,
    3, 3, 3, 3,
    4, 4, 4, 4,
    5, 5, 5,
    6, 6)
relative.frequency <- table(ideal.mana.curve) / length(ideal.mana.curve)

\end{lstlisting}

To differentiate an optimal mana curve from a suboptimal one,
let us first recall cumulative distribution functions (CDFs) from statistics.
Given a discrete random variable $ X $ and a probability mass function $ f $,
a CDF $ F $ is defined as:

$$
F(x) = P(X \le x) = \sum_{t \le x} f(t).
$$

We can use the CDF to describe our mana curve as follows:

\begin{tabular}{|l|l|}
\hline

$ F(1) $ & $ \frac{1}{19} $ \\ \hline
$ F(2) $ & $ (\frac{1}{19}) + \frac{5}{19} = \frac{6}{19} $ \\ \hline
$ F(3) $ & $ (\frac{6}{19}) + \frac{4}{19} = \frac{10}{19} $ \\ \hline
$ F(4) $ & $ (\frac{10}{19}) + \frac{4}{19} = \frac{14}{19} $ \\ \hline
$ F(5) $ & $ (\frac{14}{19}) + \frac{3}{19} = \frac{17}{19} $ \\ \hline
$ F(6) $ & $ (\frac{17}{19}) + \frac{2}{19} = \frac{19}{19} $ \\ \hline
$ F(7) $ & $ (\frac{19}{19}) + \frac{0}{19} = \frac{19}{19} $ \\ \hline
$ F(8) $ & $ (\frac{19}{19}) + \frac{0}{19} = \frac{19}{19} $ \\ \hline

\end{tabular}

Now that we have a definition of the ideal mana curve,
let us contrast it to a suboptimal mana curve
so we can develop a quantitative measure for how the two differ.
For example, take the following mana curve:


\begin{lstlisting}[language=R, frame=single]

bad.mana.curve <- c(
    1,
    2,
    3,
    4, 4, 4, 4,
    5, 5, 5, 5, 5, 5,
    6, 6,
    7, 7,
    8, 8, 8, 8, 8, 8)

bad.relative.frequency <- table(bad.mana.curve) / length(bad.mana.curve)
\end{lstlisting}


Using its CDF $ F' $, we can describe this mana curve with the following table:

\begin{tabular}{|l|l|}
\hline

$ F'(1) $ & $ \frac{1}{23} $ \\ \hline
$ F'(2) $ & $ (\frac{1}{23}) + \frac{1}{23} = \frac{2}{23} $ \\ \hline
$ F'(3) $ & $ (\frac{2}{23}) + \frac{1}{23} = \frac{3}{23} $ \\ \hline
$ F'(4) $ & $ (\frac{3}{23}) + \frac{4}{23} = \frac{7}{23} $ \\ \hline
$ F'(5) $ & $ (\frac{7}{23}) + \frac{6}{23} = \frac{13}{23} $ \\ \hline
$ F'(6) $ & $ (\frac{13}{23}) + \frac{2}{23} = \frac{15}{23} $ \\ \hline
$ F'(7) $ & $ (\frac{15}{23}) + \frac{2}{23} = \frac{17}{23} $ \\ \hline
$ F'(8) $ & $ (\frac{17}{23}) + \frac{6}{23} = \frac{23}{23} $ \\ \hline

\end{tabular}

We can define the difference between the two mana curves in the following manner:

\begin{tabular}{|l|l|}
\hline

$ \abs{F(1) - F'(1)} $ & $ \abs{(\frac{1}{19}) - (\frac{1}{23})} = \frac{4}{437} \approx 0.92\% $ \\ \hline
$ \abs{F(2) - F'(2)} $ & $ \abs{(\frac{6}{19}) - (\frac{2}{23})} = \frac{100}{437} \approx 23\% $ \\ \hline
$ \abs{F(3) - F'(3)} $ & $ \abs{(\frac{10}{19}) - (\frac{3}{23})} = \frac{173}{437} \approx 40\% $ \\ \hline
$ \abs{F(4) - F'(4)} $ & $ \abs{(\frac{14}{19}) - (\frac{7}{23})} = \frac{189}{437} \approx 43\% $ \\ \hline
$ \abs{F(5) - F'(5)} $ & $ \abs{(\frac{17}{19}) - (\frac{13}{23})} = \frac{144}{437} \approx 33\% $ \\ \hline
$ \abs{F(6) - F'(6)} $ & $ \abs{(\frac{19}{19}) - (\frac{15}{23})} = \frac{8}{23} \approx 35\% $ \\ \hline
$ \abs{F(7) - F'(7)} $ & $ \abs{(\frac{19}{19}) - (\frac{17}{23})} = \frac{6}{23} \approx 26\% $ \\ \hline
$ \abs{F(8) - F'(8)} $ & $ \abs{(\frac{19}{19}) - (\frac{23}{23})} = 0\% $ \\ \hline

\end{tabular}

Using this approach,
we can define a penalty for being different from our ideal mana curve.
Namely,
\scalebox{0.5}{
$$
penalty(F') = \abs{5\% - F'(1)} + \abs{31\% - F'(2)} + \abs{52\% - F'(3)}
    + \abs{73\% - F'(4)} + \abs{89\% - F'(5)} + \abs{100\% - F'(6)}
    + \abs{100\% - F'(7)} + \abs{100\% - F'(8)}
$$}

Using this function, the penalty for the suboptimal mana curve would be
$ penalty(F') \approx 2.00 $.

\subsubsection{Land Percentage}
\label{objective:Land Percentage}

During gameplay,
players pay the cost for these cards in terms of a special card type known as a land.
Lands have a 0 CMC (so they are free to play)
and have a specific color corresponding to the kind of mana cost they can satisfy.
As with the previous sub-objective,
there exists another fine balance between having powerful non-land cards and lands to support them.
Having too few lands means not having enough mana to play the non-land cards
but, at the same time, having too many lands means not having enough non-land cards to play.
Experts recommend having a 17:23 lands-to-non-land ratio in a 40-card deck,
which is about 42.5\% lands,
but also permit 16:24 and 18:22 ratios.

In either case,
we want this ratio to be below 50\% since if it were 50\%,
then there would be a 50/50 chance every time the player draws a card that the card would be a land.
A ratio higher than 50\% would yield even more lands.
This many lands is not practical in-game,
so we want the ratio to be lower.
We will calculate the penalty for having a higher
or lower ratio of lands to non-lands with the following function:

$$
f(r) = \begin{cases}
    0 & \frac{16}{40} <= r <= \frac{18}{40} \\
    \abs{\frac{17}{40} - r} & otherwise
\end{cases}
$$

with $ r $ being our ratio of lands to non-lands.

\subsubsection{Land Color Percentage}
\label{objective:Land Color Percentage}

As mentioned previously,
most cards in Magic: The Gathering require mana in order to be played
and the mana cost is denoted with mana symbols in the top-right corner.
Some mana symbols require a specific color of mana, such as black or red,
whereas others accept any color.
We want to craft the color ratio of our lands such that it is equal to
the ratio of the mana cost of the cards in our deck.
For example,
if we take the cards in our deck and count all the red mana symbols on them,
we can find the ratio of red mana symbols to the total number of mana symbols
and we should strive to make ratio of our red lands to the total number of lands equal to it.

Say we count 19 red mana symbols of 35 mana symbols in total.
The ratio of red mana symbols is $ \frac{19}{35} $ or about $ 54\% $.
Lets say that in this example we want to have 16 lands total.
Given this ratio,
we would want $ \frac{19}{35} * 16 \approx 9 $ of our lands to be red lands.

By comparing the deck being analyzed's ratio of each land and the total number of lands
to the ratio of mana symbols of the color that land would produce and the total number of mana symbols,
we can analyze how ideal our landbase is. For each color of mana symbol used within the deck,
we apply this function to calculate the penalty for the incorrectness of the manabase

$f(w,u,b,r,g)=|w'-w|+|u'-u|+|b'-b|+|r'-r|+|g'-g|$

where w, u, b, r, and g denote the land color ratio for each color within the deck
while their prime counterparts represent the corresponding mana symbol ratio for the deck.

\subsubsection{Color Identity}
\label{objective:Color Identity}

The ideal deck primarily consists of two mana colors.
Any other color underneath a certain threshold is considered to be a \enquote{splash}.
We define a splash to be a color with a relative frequency less than 10\%.
The best case scenario is a deck that is solely two color deck, with no splashes,
so our function will penalize any colors that do not count as the dominant two.
% It will also make sure that if a splash does occur, it only occurs once.
The penalty would be determined by a linear function such as:

$$
f(colors, splashes) = c_1 * \abs{2 - colors} + c_2 * splashes
$$

where `colors' denotes the number of dominant colors,
`splashes' denotes the number of splashed colors,
and $ c_1 $ \& $ c_2 $ are separate values to weight the results.

\subsubsection{Card Archetypes}
\label{objective:Card Archetypes}

In the card game Magic: The Gathering there are many \enquote{archetypes} of cards that are
important to keep in mind when creating a sealed deck.
The acronym BREAD is used to describe some of these archetypes.
This acronym stands for Bombs, Removal, Evasion, Aggro, and Duds.
In addition to the archetypes described,
I believe that there are other archetypes such as
mana-fixing and combat tricks,
which are also key in creating a successful sealed Magic: The Gathering deck.

Bombs refer to cards with such powerful effects or strength that they can potentially turn
the tide of the game in your favor just by playing them.
In a general context, we would ideally have one or more bombs.
So, in calculating the score of the deck,
if the deck contains a bomb
then no deduction will be made in this regard.
Otherwise, there will be a deduction of $ 10 $.
The penalty for bombs can be represented as:

$$
b(bombs) = \begin{cases}
    0 & bombs \ge 1 \\
    10 & otherwise
\end{cases}
$$

where `bombs' denotes the number of bombs present within the deck.

\enquote{Removals} are
cards that remove your opponent's cards through mechanisms such as
damage, destruction, removal from the game, or preventing your opponent from even playing the threat.
In the context of our chosen tournament type,
removals are those that generally deal damage to creatures,
as access to good
artifacts, enchantments, and planeswalkers tend to be very few and far between.
As this category is a key element in many successful limited decks,
we want at least $ 2 $ removal spells to stop our opponent from progressing their own board state.
The penalty for removal can be represented as

$$
r(removal) = \begin{cases}
    0 & removal \ge 2 \\
    c(2-removal) & otherwise
\end{cases}
$$

where `removal' denotes the number of removal spells present within the deck
and $ c $ denoting a constant to weigh the result.

\enquote{Evasive} cards are
more difficult for an opponent to block.
They require the opponent to use a removal spell or have creatures that meet a specific criterion.
Some examples of evasive cards are creatures
with flying or that are unblockable by other means.
Evasion is important as it allows you to damage an opponent
in a fashion that he would otherwise not have to deal with.
Since evasive creatures do not always present in every color,
we will only penalize decks whose dominant colors contain the possibility for an evasive.
Only if an evasive card in one of those colors is provided as an option,
will we penalize the deck if it does not include at least $ 2 $ of them.
The penalty for removal can be represented as

$$
e(evasive) = \begin{cases}
    c(2 - evasive) & evasive < 2 \land \text{evasive is an option in the deck's dominant colors} \\
    0 & otherwise
\end{cases}
$$

where `evasive' denotes the number of evasive creatures present within the deck
and $ c $ denoting a constant to weigh the result.

\enquote{Aggro} is the archetype describing aggressive creatures or spells,
which limited decks tend to be primarily comprised of.
These cards allow for constant pressure to be placed upon opponents
and potentially allow you to win the game.
They do not have the sheer strength or board presence of a bomb
but instead act as smaller, yet constant, nuisances that your opponent
has to handle or risk losing the game.
As this is a good portion of limited decks,
the quantity of these types of cards will likely be greater than those of other archetypes.

When creating a sealed deck,
there will ultimately be cards that are in the pool of cards
that can be potentially put into a deck that will be near unusable due to multiple reasons.
there can be cards that are nearly unusable for multiple reasons.
Some of these reasons include cards of different colors or
cards that are simply just too bad to utilize.
These cards are of the final letter in the BREAD acronym called \enquote{duds}.
In a sealed deck,
card quality is very important to the success of the deck,
so we want to prioritize the better cards to increase its quality.
Decks containing these \enquote{duds} will receive a penalty of $ 5 $ per dud card
as they are virtually unplayable.
The penalty for including duds within a deck can be represented as

$$
d(duds) = 5 * duds
$$

where `duds' denotes the number of dud cards within the deck.

In a game of Magic: The Gathering,
in order to cast a spell one needs access to certain colors of mana to pay for the cost of each spell.
Since there are 5 different colors of mana,
the more colors that one chooses for their deck to contain
then the harder it will be to obtain all of the mana in the right color combinations
in order for them to cast their spells.
Mana-fixing cards allow for one to usually choose one of multiple colors of mana
to allow for flexibility in how they play their spells.
If a deck is running multiple colors
it is highly recommended to use cards that allow for you to fix your
mana supply with the colors that you need.
In the case that a play decides to play more than two colors,
they almost need to have mana fixing
otherwise in some games they will just sit there hoping
they get the land to provide them with the
mana they need to do anything, putting them at a severe disadvantage.
If a deck is playing 2 colors then usually only 1
mana-fixer is needed but any more than two colors and you definitely want two or three.
The deduction due to mana-fixing would be if a three or more color deck
has less than $ 2 $ mana-fixers then the deduction would be based on the difference of
2 and the number of mana-fixers that reside within the deck.
The mana-fixing deduction can be represented as

$$
m(mana) = \begin{cases}
    0 & x \ge 2 \\
    c(2 - mana) & otherwise
\end{cases}
$$

where `mana' denotes the number of mana-fixers within the deck
and $ c $ being a constant to weight the results.

During combat in Magic: The Gathering
the active player chooses which of their creatures to attack with
and then the defending player decides how they would like to block the attacking creatures.
Then the attacking creatures deal damage to the blocking creatures and vice-versa.
By a creature reaching 0 toughness or due to another ability,
creatures die and enter the graveyard where they usually lie until the end of the game.
By killing off your opponent's creatures,
you usually are given an advantage.
One way of accomplishing this is by using \enquote{combat tricks}.
Since your opponent does not know the cards in your hand,
they are unaware of what your plans are.
Combat tricks are cards or abilities that allow for interaction
with the attacking or blocking creatures.
This allows you to mess up your opponent's plans in your favor.

To calculate the complete value of penalties within this subsection,
we take the sum of all of the deductions made,
represented by:

$$
TP(deck) = b(bombs) + r(removal) + e(evasive) + d(duds) + m(mana)
$$

\subsection{Optimization}
\label{sec:Optimization}


Through our research, we have found three possible optimization algorithms
that we could use to select the cards for our objective function to grade.
These are ant colony optimization, particle swarm optimization,
and compartmentalized knapsack optimization.

\phantomsection
\label{paragraph:Ant colony}
Ant colony optimization is based off of the way ants move around.
When an ant finds a path leading to food,
they release a pheromone to attract other ants to tell they are headed the right way.
In this optimization algorithm, artificial `ants' move through a tree
containing all possible solutions to locate the optimal ones.
The ants record their positions and the quality of the solution to direct each other.
Since we can evaluate how good each card with the objective function described above,
the use of this algorithm
would allow us to find the best cards to use from the pool.

\phantomsection
\label{paragraph:Particle Swarm}
The second optimization algorithm we are considering is particle swarm optimization.
This method simulates birds flocking to food.
It works by having all particles follow the two \enquote{best} values.
These two bests are the local best and the global best.
After each iteration the velocities of each particle is updated.
While this form of optimization may not be the best in its normal state,
if we can modify it, this method would be preferable.
This is because we would be able to account for card synergies.
This would allow for the best deck to be selected given the cards
in the pool based off of more situational factors.

The third algorithm is a variation of the knapsack algorithm discussed during class.
It uses the same idea of maximizing the value of a \enquote{knapsack,}
which contains some elements.
This variation stores elements of the same type in compartments within the knapsack.
The compartments each have flexible capacities, but are lower and upper bounded.
The goal is to maximize the total value of all of the items in the knapsack
minus the cost of the compartments.


While searching for an efficient optimization algorithm that would suit our needs,
we discovered a project similar to ours that a grad student
named Troy Hernandez developed while working toward's his Master's degree.
This project took the cards from the booster packs the player's received
and returned the highest rated deck.
Similar to our project, though our project can currently be used for any Magic: The Gathering
Ravnica Allegiance deck, while Hernandez's project can only be used in the sealed format.
Our project could be modified to accept cards from any Magic: The Gathering set,
given more time.
Hernandez's optimization algorithm was written entirely in R,
and while we searched for a method of integrating an R function
into Python, we ended up not going with this method.


\section{Contributions}

\begin{itemize}

\item Scott Harris

\begin{itemize}

    \item Literary Review Sections \ref{sec:3.1} \& \ref{sec:3.2}
    \item Archetypes Sub-objective (\ref{objective:Card Archetypes})
    \item Magic: The Gathering Consultant/Researcher
    \item Contribution \%: 30\%

\end{itemize}

\item John Meyer

\begin{itemize}

    \item Literary Review Sections \ref{sec:3.5} \& \ref{sec:3.6}
    \item Mana-Curve Sub-objective (\ref{objective:Mana Curve})
    \item Project Lead
    \item Contribution \%: 40\%

\end{itemize}

\item Bryan Hayes

\begin{itemize}

    \item Literary Review Sections \ref{sec:3.7} \& \ref{sec:3.8}
    \item Deck-size (\ref{objective:Deck Size}) \&
        Land Ratio (\ref{objective:Land Percentage}, \ref{objective:Land Color Percentage}) Sub-Objectives
    \item Ant Colony Optimization Paragraph (\ref{paragraph:Ant colony})
    \item Contribution \%: 15\%

\end{itemize}

\item Jacob Freedman

\begin{itemize}

    \item Literary Review Sections \ref{sec:3.3} \& \ref{sec:3.4}
    \item Color Identity (\ref{objective:Color Identity}) Sub-objective
    \item Researched Optimization
    \item Particle Swarm Optimization Paragraph (\ref{paragraph:Particle Swarm})
    \item Contribution \%: 15\%

\end{itemize}

\end{itemize}

\begin{thebibliography}{9}

% Scott's

\bibitem{Cowling2012}
Peter I. Cowling, Colin D. Ward, and Edward J. Powley. 2012.
\enquote{Ensemble Determinization in Monte Carlo Tree Search for the Imperfect Information Card Game Magic: The
Gathering.}
(June 2012). Retrieved March 4, 2019 from https://ieeexplore.ieee.org/document/6218176/

\bibitem{Ham2010}
Ethan Ham. 2010. (April 2010).
\enquote{Rarity and Power: Balance in Collectible Object Games.}
Retrieved March 4, 2019 from http://gamestudies.org/1001/articles/ham

% Jacob's

\bibitem{Pelikan1999}
Martin Pelikan, David Goldburg, Erick Cantú-Paz. 1999. BOA: Bayesian Optimization Algorithm. Illinois Genetic Algorithms
Laboratory. University of Illinois at Urbana-Champaign,Urbana IL.

\bibitem{Saint-Etienne2018}
Marie-Liesse Cauwet Mines Saint-Etienne, Olivier Teytaud. 2018. Surprising strategies obtained by stochastic
optimization in partially observable games. Univ Clermont Auvergne, CNRS, UMR 6158 LIMOS, Institut Henri Fayol,
Departement GMI, France

% John's:

\bibitem{Buchheim2014}
% ACM citation style
Christoph Buchheim and Long Trieu. 2014.
Active Set Methods with Reoptimization for Convex Quadratic Integer Programming.
In \textit{Combinatorial Optimization: Third International Symposium,
ISCO 2014 Lisbon, Portugal, March 5–7, 2014 Revised Selected Papers}.
Springer International Publishing Switzerland.
DOI: 10.1007/978-3-319-09174-7.

\bibitem{Furer2014}
% ACM citation style
Martin Fürer and Huiwen Yu. 2014.
Approximating the k-Set Packing Problem by Local Improvements.
In \textit{Combinatorial Optimization: Third International Symposium,
ISCO 2014 Lisbon, Portugal, March 5–7, 2014 Revised Selected Papers}.
Springer International Publishing Switzerland.
DOI: 10.1007/978-3-319-09174-7.

% Bryan's:

\bibitem{Chatterjee}
Krishnendu Chatterjee and Rasmus Ibsen-Jensen.
The Complexity of Deciding Legality of a Single Step of Magic: The Gathering.

\bibitem{Crutcher2017}
Paul A. Crutcher. 2017. Magic: The Gathering, A Literary Text.
\textit{Americana: The Journal of American Popular Culture} 16, 1 (2017).

\end{thebibliography}

\end{document}
